\documentclass[]{article}
\usepackage{lmodern}
\usepackage{amssymb,amsmath}
\usepackage{ifxetex,ifluatex}
\usepackage{fixltx2e} % provides \textsubscript
\ifnum 0\ifxetex 1\fi\ifluatex 1\fi=0 % if pdftex
  \usepackage[T1]{fontenc}
  \usepackage[utf8]{inputenc}
\else % if luatex or xelatex
  \ifxetex
    \usepackage{mathspec}
  \else
    \usepackage{fontspec}
  \fi
  \defaultfontfeatures{Ligatures=TeX,Scale=MatchLowercase}
\fi
% use upquote if available, for straight quotes in verbatim environments
\IfFileExists{upquote.sty}{\usepackage{upquote}}{}
% use microtype if available
\IfFileExists{microtype.sty}{%
\usepackage{microtype}
\UseMicrotypeSet[protrusion]{basicmath} % disable protrusion for tt fonts
}{}
\usepackage[margin=1in]{geometry}
\usepackage{hyperref}
\hypersetup{unicode=true,
            pdftitle={ReproducibleResearchAssgmt2},
            pdfauthor={majusus},
            pdfborder={0 0 0},
            breaklinks=true}
\urlstyle{same}  % don't use monospace font for urls
\usepackage{color}
\usepackage{fancyvrb}
\newcommand{\VerbBar}{|}
\newcommand{\VERB}{\Verb[commandchars=\\\{\}]}
\DefineVerbatimEnvironment{Highlighting}{Verbatim}{commandchars=\\\{\}}
% Add ',fontsize=\small' for more characters per line
\usepackage{framed}
\definecolor{shadecolor}{RGB}{248,248,248}
\newenvironment{Shaded}{\begin{snugshade}}{\end{snugshade}}
\newcommand{\KeywordTok}[1]{\textcolor[rgb]{0.13,0.29,0.53}{\textbf{#1}}}
\newcommand{\DataTypeTok}[1]{\textcolor[rgb]{0.13,0.29,0.53}{#1}}
\newcommand{\DecValTok}[1]{\textcolor[rgb]{0.00,0.00,0.81}{#1}}
\newcommand{\BaseNTok}[1]{\textcolor[rgb]{0.00,0.00,0.81}{#1}}
\newcommand{\FloatTok}[1]{\textcolor[rgb]{0.00,0.00,0.81}{#1}}
\newcommand{\ConstantTok}[1]{\textcolor[rgb]{0.00,0.00,0.00}{#1}}
\newcommand{\CharTok}[1]{\textcolor[rgb]{0.31,0.60,0.02}{#1}}
\newcommand{\SpecialCharTok}[1]{\textcolor[rgb]{0.00,0.00,0.00}{#1}}
\newcommand{\StringTok}[1]{\textcolor[rgb]{0.31,0.60,0.02}{#1}}
\newcommand{\VerbatimStringTok}[1]{\textcolor[rgb]{0.31,0.60,0.02}{#1}}
\newcommand{\SpecialStringTok}[1]{\textcolor[rgb]{0.31,0.60,0.02}{#1}}
\newcommand{\ImportTok}[1]{#1}
\newcommand{\CommentTok}[1]{\textcolor[rgb]{0.56,0.35,0.01}{\textit{#1}}}
\newcommand{\DocumentationTok}[1]{\textcolor[rgb]{0.56,0.35,0.01}{\textbf{\textit{#1}}}}
\newcommand{\AnnotationTok}[1]{\textcolor[rgb]{0.56,0.35,0.01}{\textbf{\textit{#1}}}}
\newcommand{\CommentVarTok}[1]{\textcolor[rgb]{0.56,0.35,0.01}{\textbf{\textit{#1}}}}
\newcommand{\OtherTok}[1]{\textcolor[rgb]{0.56,0.35,0.01}{#1}}
\newcommand{\FunctionTok}[1]{\textcolor[rgb]{0.00,0.00,0.00}{#1}}
\newcommand{\VariableTok}[1]{\textcolor[rgb]{0.00,0.00,0.00}{#1}}
\newcommand{\ControlFlowTok}[1]{\textcolor[rgb]{0.13,0.29,0.53}{\textbf{#1}}}
\newcommand{\OperatorTok}[1]{\textcolor[rgb]{0.81,0.36,0.00}{\textbf{#1}}}
\newcommand{\BuiltInTok}[1]{#1}
\newcommand{\ExtensionTok}[1]{#1}
\newcommand{\PreprocessorTok}[1]{\textcolor[rgb]{0.56,0.35,0.01}{\textit{#1}}}
\newcommand{\AttributeTok}[1]{\textcolor[rgb]{0.77,0.63,0.00}{#1}}
\newcommand{\RegionMarkerTok}[1]{#1}
\newcommand{\InformationTok}[1]{\textcolor[rgb]{0.56,0.35,0.01}{\textbf{\textit{#1}}}}
\newcommand{\WarningTok}[1]{\textcolor[rgb]{0.56,0.35,0.01}{\textbf{\textit{#1}}}}
\newcommand{\AlertTok}[1]{\textcolor[rgb]{0.94,0.16,0.16}{#1}}
\newcommand{\ErrorTok}[1]{\textcolor[rgb]{0.64,0.00,0.00}{\textbf{#1}}}
\newcommand{\NormalTok}[1]{#1}
\usepackage{graphicx,grffile}
\makeatletter
\def\maxwidth{\ifdim\Gin@nat@width>\linewidth\linewidth\else\Gin@nat@width\fi}
\def\maxheight{\ifdim\Gin@nat@height>\textheight\textheight\else\Gin@nat@height\fi}
\makeatother
% Scale images if necessary, so that they will not overflow the page
% margins by default, and it is still possible to overwrite the defaults
% using explicit options in \includegraphics[width, height, ...]{}
\setkeys{Gin}{width=\maxwidth,height=\maxheight,keepaspectratio}
\IfFileExists{parskip.sty}{%
\usepackage{parskip}
}{% else
\setlength{\parindent}{0pt}
\setlength{\parskip}{6pt plus 2pt minus 1pt}
}
\setlength{\emergencystretch}{3em}  % prevent overfull lines
\providecommand{\tightlist}{%
  \setlength{\itemsep}{0pt}\setlength{\parskip}{0pt}}
\setcounter{secnumdepth}{0}
% Redefines (sub)paragraphs to behave more like sections
\ifx\paragraph\undefined\else
\let\oldparagraph\paragraph
\renewcommand{\paragraph}[1]{\oldparagraph{#1}\mbox{}}
\fi
\ifx\subparagraph\undefined\else
\let\oldsubparagraph\subparagraph
\renewcommand{\subparagraph}[1]{\oldsubparagraph{#1}\mbox{}}
\fi

%%% Use protect on footnotes to avoid problems with footnotes in titles
\let\rmarkdownfootnote\footnote%
\def\footnote{\protect\rmarkdownfootnote}

%%% Change title format to be more compact
\usepackage{titling}

% Create subtitle command for use in maketitle
\newcommand{\subtitle}[1]{
  \posttitle{
    \begin{center}\large#1\end{center}
    }
}

\setlength{\droptitle}{-2em}

  \title{ReproducibleResearchAssgmt2}
    \pretitle{\vspace{\droptitle}\centering\huge}
  \posttitle{\par}
    \author{majusus}
    \preauthor{\centering\large\emph}
  \postauthor{\par}
      \predate{\centering\large\emph}
  \postdate{\par}
    \date{December 9, 2018}


\begin{document}
\maketitle

\subsection{1: Synopsis}\label{synopsis}

The goal of the assignment is to explore the NOAA Storm Database and
explore the effects of severe weather events on both population and
economy.The database covers the time period between 1950 and November
2011.

The following analysis investigates which types of severe weather events
are most harmful on:

\begin{enumerate}
\def\labelenumi{\arabic{enumi}.}
\tightlist
\item
  Health (injuries and fatalities)
\item
  Property and crops (economic consequences)
\end{enumerate}

Information on the Data:
\href{https://d396qusza40orc.cloudfront.net/repdata\%2Fpeer2_doc\%2Fpd01016005curr.pdf}{Documentation}

\subsection{2: Data Processing}\label{data-processing}

\subsubsection{2.1: Data Loading}\label{data-loading}

Download the raw data file and extract the data into a dataframe.Then
convert to a data.table

\begin{Shaded}
\begin{Highlighting}[]
\KeywordTok{library}\NormalTok{(}\StringTok{"data.table"}\NormalTok{)}
\KeywordTok{library}\NormalTok{(}\StringTok{"ggplot2"}\NormalTok{)}
\NormalTok{fileUrl <-}\StringTok{ "https://d396qusza40orc.cloudfront.net/repdata%2Fdata%2FStormData.csv.bz2"}
\KeywordTok{download.file}\NormalTok{(fileUrl, }\DataTypeTok{destfile =} \KeywordTok{paste0}\NormalTok{(}\StringTok{"E:/datascience/datasciencecoursera/Course5Assignment2"}\NormalTok{, }\StringTok{'/repdata%2Fdata%2FStormData.csv.bz2'}\NormalTok{))}
\NormalTok{stormDF <-}\StringTok{ }\KeywordTok{read.csv}\NormalTok{(}\StringTok{"E:/datascience/datasciencecoursera/Course5Assignment2/repdata%2Fdata%2FStormData.csv.bz2"}\NormalTok{)}
\CommentTok{# Converting data.frame to data.table}
\NormalTok{stormDT <-}\StringTok{ }\KeywordTok{as.data.table}\NormalTok{(stormDF)}
\end{Highlighting}
\end{Shaded}

\subsubsection{2.2: Examining Column
Names}\label{examining-column-names}

\begin{Shaded}
\begin{Highlighting}[]
\KeywordTok{colnames}\NormalTok{(stormDT)}
\end{Highlighting}
\end{Shaded}

\begin{verbatim}
##  [1] "STATE__"    "BGN_DATE"   "BGN_TIME"   "TIME_ZONE"  "COUNTY"    
##  [6] "COUNTYNAME" "STATE"      "EVTYPE"     "BGN_RANGE"  "BGN_AZI"   
## [11] "BGN_LOCATI" "END_DATE"   "END_TIME"   "COUNTY_END" "COUNTYENDN"
## [16] "END_RANGE"  "END_AZI"    "END_LOCATI" "LENGTH"     "WIDTH"     
## [21] "F"          "MAG"        "FATALITIES" "INJURIES"   "PROPDMG"   
## [26] "PROPDMGEXP" "CROPDMG"    "CROPDMGEXP" "WFO"        "STATEOFFIC"
## [31] "ZONENAMES"  "LATITUDE"   "LONGITUDE"  "LATITUDE_E" "LONGITUDE_"
## [36] "REMARKS"    "REFNUM"
\end{verbatim}

\subsubsection{2.3: Data Subsetting}\label{data-subsetting}

Subset the dataset on the parameters of interest. Basically, we remove
the columns we don't need for clarity.

\begin{Shaded}
\begin{Highlighting}[]
\CommentTok{# Finding columns to remove}
\NormalTok{cols2Remove <-}\StringTok{ }\KeywordTok{colnames}\NormalTok{(stormDT[, }\OperatorTok{!}\KeywordTok{c}\NormalTok{(}\StringTok{"EVTYPE"}
\NormalTok{  , }\StringTok{"FATALITIES"}
\NormalTok{  , }\StringTok{"INJURIES"}
\NormalTok{  , }\StringTok{"PROPDMG"}
\NormalTok{  , }\StringTok{"PROPDMGEXP"}
\NormalTok{  , }\StringTok{"CROPDMG"}
\NormalTok{  , }\StringTok{"CROPDMGEXP"}\NormalTok{)])}
\CommentTok{# Removing columns}
\NormalTok{stormDT[, }\KeywordTok{c}\NormalTok{(cols2Remove) }\OperatorTok{:}\ErrorTok{=}\StringTok{ }\OtherTok{NULL}\NormalTok{]}
\CommentTok{# Only use data where fatalities or injuries occurred.  }
\NormalTok{stormDT <-}\StringTok{ }\NormalTok{stormDT[(EVTYPE }\OperatorTok{!=}\StringTok{ "?"} \OperatorTok{&}\StringTok{ }
\StringTok{             }\NormalTok{(INJURIES }\OperatorTok{>}\StringTok{ }\DecValTok{0} \OperatorTok{|}\StringTok{ }\NormalTok{FATALITIES }\OperatorTok{>}\StringTok{ }\DecValTok{0} \OperatorTok{|}\StringTok{ }\NormalTok{PROPDMG }\OperatorTok{>}\StringTok{ }\DecValTok{0} \OperatorTok{|}\StringTok{ }\NormalTok{CROPDMG }\OperatorTok{>}\StringTok{ }\DecValTok{0}\NormalTok{)), }\KeywordTok{c}\NormalTok{(}\StringTok{"EVTYPE"}
\NormalTok{                                                                            , }\StringTok{"FATALITIES"}
\NormalTok{                                                                            , }\StringTok{"INJURIES"}
\NormalTok{                                                                            , }\StringTok{"PROPDMG"}
\NormalTok{                                                                            , }\StringTok{"PROPDMGEXP"}
\NormalTok{                                                                            , }\StringTok{"CROPDMG"}
\NormalTok{                                                                            , }\StringTok{"CROPDMGEXP"}\NormalTok{) ]}
\end{Highlighting}
\end{Shaded}

\subsubsection{2.4: Converting Exponent Columns into Actual Exponents
instead of (-,+, H, K,
etc)}\label{converting-exponent-columns-into-actual-exponents-instead-of---h-k-etc}

Making the PROPDMGEXP and CROPDMGEXP columns cleaner so they can be used
to calculate property and crop cost.

\begin{Shaded}
\begin{Highlighting}[]
\CommentTok{# Change all damage exponents to uppercase.}
\NormalTok{cols <-}\StringTok{ }\KeywordTok{c}\NormalTok{(}\StringTok{"PROPDMGEXP"}\NormalTok{, }\StringTok{"CROPDMGEXP"}\NormalTok{)}
\NormalTok{stormDT[,  (cols) }\OperatorTok{:}\ErrorTok{=}\StringTok{ }\KeywordTok{c}\NormalTok{(}\KeywordTok{lapply}\NormalTok{(.SD, toupper)), .SDcols =}\StringTok{ }\NormalTok{cols]}
\CommentTok{# Map property damage alphanumeric exponents to numeric values.}
\NormalTok{propDmgKey <-}\StringTok{  }\KeywordTok{c}\NormalTok{(}\StringTok{"}\CharTok{\textbackslash{}"\textbackslash{}"}\StringTok{"}\NormalTok{ =}\StringTok{ }\DecValTok{10}\OperatorTok{^}\DecValTok{0}\NormalTok{,}
                 \StringTok{"-"}\NormalTok{ =}\StringTok{ }\DecValTok{10}\OperatorTok{^}\DecValTok{0}\NormalTok{, }
                 \StringTok{"+"}\NormalTok{ =}\StringTok{ }\DecValTok{10}\OperatorTok{^}\DecValTok{0}\NormalTok{,}
                 \StringTok{"0"}\NormalTok{ =}\StringTok{ }\DecValTok{10}\OperatorTok{^}\DecValTok{0}\NormalTok{,}
                 \StringTok{"1"}\NormalTok{ =}\StringTok{ }\DecValTok{10}\OperatorTok{^}\DecValTok{1}\NormalTok{,}
                 \StringTok{"2"}\NormalTok{ =}\StringTok{ }\DecValTok{10}\OperatorTok{^}\DecValTok{2}\NormalTok{,}
                 \StringTok{"3"}\NormalTok{ =}\StringTok{ }\DecValTok{10}\OperatorTok{^}\DecValTok{3}\NormalTok{,}
                 \StringTok{"4"}\NormalTok{ =}\StringTok{ }\DecValTok{10}\OperatorTok{^}\DecValTok{4}\NormalTok{,}
                 \StringTok{"5"}\NormalTok{ =}\StringTok{ }\DecValTok{10}\OperatorTok{^}\DecValTok{5}\NormalTok{,}
                 \StringTok{"6"}\NormalTok{ =}\StringTok{ }\DecValTok{10}\OperatorTok{^}\DecValTok{6}\NormalTok{,}
                 \StringTok{"7"}\NormalTok{ =}\StringTok{ }\DecValTok{10}\OperatorTok{^}\DecValTok{7}\NormalTok{,}
                 \StringTok{"8"}\NormalTok{ =}\StringTok{ }\DecValTok{10}\OperatorTok{^}\DecValTok{8}\NormalTok{,}
                 \StringTok{"9"}\NormalTok{ =}\StringTok{ }\DecValTok{10}\OperatorTok{^}\DecValTok{9}\NormalTok{,}
                 \StringTok{"H"}\NormalTok{ =}\StringTok{ }\DecValTok{10}\OperatorTok{^}\DecValTok{2}\NormalTok{,}
                 \StringTok{"K"}\NormalTok{ =}\StringTok{ }\DecValTok{10}\OperatorTok{^}\DecValTok{3}\NormalTok{,}
                 \StringTok{"M"}\NormalTok{ =}\StringTok{ }\DecValTok{10}\OperatorTok{^}\DecValTok{6}\NormalTok{,}
                 \StringTok{"B"}\NormalTok{ =}\StringTok{ }\DecValTok{10}\OperatorTok{^}\DecValTok{9}\NormalTok{)}
\CommentTok{# Map crop damage alphanumeric exponents to numeric values}
\NormalTok{cropDmgKey <-}\StringTok{  }\KeywordTok{c}\NormalTok{(}\StringTok{"}\CharTok{\textbackslash{}"\textbackslash{}"}\StringTok{"}\NormalTok{ =}\StringTok{ }\DecValTok{10}\OperatorTok{^}\DecValTok{0}\NormalTok{,}
                \StringTok{"?"}\NormalTok{ =}\StringTok{ }\DecValTok{10}\OperatorTok{^}\DecValTok{0}\NormalTok{, }
                \StringTok{"0"}\NormalTok{ =}\StringTok{ }\DecValTok{10}\OperatorTok{^}\DecValTok{0}\NormalTok{,}
                \StringTok{"K"}\NormalTok{ =}\StringTok{ }\DecValTok{10}\OperatorTok{^}\DecValTok{3}\NormalTok{,}
                \StringTok{"M"}\NormalTok{ =}\StringTok{ }\DecValTok{10}\OperatorTok{^}\DecValTok{6}\NormalTok{,}
                \StringTok{"B"}\NormalTok{ =}\StringTok{ }\DecValTok{10}\OperatorTok{^}\DecValTok{9}\NormalTok{)}
\NormalTok{stormDT[, PROPDMGEXP }\OperatorTok{:}\ErrorTok{=}\StringTok{ }\NormalTok{propDmgKey[}\KeywordTok{as.character}\NormalTok{(stormDT[,PROPDMGEXP])]]}
\NormalTok{stormDT[}\KeywordTok{is.na}\NormalTok{(PROPDMGEXP), PROPDMGEXP }\OperatorTok{:}\ErrorTok{=}\StringTok{ }\DecValTok{10}\OperatorTok{^}\DecValTok{0}\NormalTok{ ]}
\NormalTok{stormDT[, CROPDMGEXP }\OperatorTok{:}\ErrorTok{=}\StringTok{ }\NormalTok{cropDmgKey[}\KeywordTok{as.character}\NormalTok{(stormDT[,CROPDMGEXP])] ]}
\NormalTok{stormDT[}\KeywordTok{is.na}\NormalTok{(CROPDMGEXP), CROPDMGEXP }\OperatorTok{:}\ErrorTok{=}\StringTok{ }\DecValTok{10}\OperatorTok{^}\DecValTok{0}\NormalTok{ ]}
\end{Highlighting}
\end{Shaded}

\subsubsection{2.5: Making Economic Cost
Columns}\label{making-economic-cost-columns}

\begin{Shaded}
\begin{Highlighting}[]
\NormalTok{stormDT <-}\StringTok{ }\NormalTok{stormDT[, .(EVTYPE, FATALITIES, INJURIES, PROPDMG, PROPDMGEXP, }\DataTypeTok{propCost =}\NormalTok{ PROPDMG }\OperatorTok{*}\StringTok{ }\NormalTok{PROPDMGEXP, CROPDMG, CROPDMGEXP, }\DataTypeTok{cropCost =}\NormalTok{ CROPDMG }\OperatorTok{*}\StringTok{ }\NormalTok{CROPDMGEXP)]}
\end{Highlighting}
\end{Shaded}

\subsubsection{2.6: Calcuating Total Property and Crop
Cost}\label{calcuating-total-property-and-crop-cost}

\begin{Shaded}
\begin{Highlighting}[]
\NormalTok{totalCostDT <-}\StringTok{ }\NormalTok{stormDT[, .(}\DataTypeTok{propCost =} \KeywordTok{sum}\NormalTok{(propCost), }\DataTypeTok{cropCost =} \KeywordTok{sum}\NormalTok{(cropCost), }\DataTypeTok{Total_Cost =} \KeywordTok{sum}\NormalTok{(propCost) }\OperatorTok{+}\StringTok{ }\KeywordTok{sum}\NormalTok{(cropCost)), by =}\StringTok{ }\NormalTok{.(EVTYPE)]}
\NormalTok{totalCostDT <-}\StringTok{ }\NormalTok{totalCostDT[}\KeywordTok{order}\NormalTok{(}\OperatorTok{-}\NormalTok{Total_Cost), ]}
\NormalTok{totalCostDT <-}\StringTok{ }\NormalTok{totalCostDT[}\DecValTok{1}\OperatorTok{:}\DecValTok{10}\NormalTok{, ]}
\KeywordTok{head}\NormalTok{(totalCostDT, }\DecValTok{5}\NormalTok{)}
\end{Highlighting}
\end{Shaded}

\begin{verbatim}
##               EVTYPE     propCost   cropCost   Total_Cost
## 1:             FLOOD 144657709807 5661968450 150319678257
## 2: HURRICANE/TYPHOON  69305840000 2607872800  71913712800
## 3:           TORNADO  56947380677  414953270  57362333947
## 4:       STORM SURGE  43323536000       5000  43323541000
## 5:              HAIL  15735267513 3025954473  18761221986
\end{verbatim}

\subsubsection{2.7: Calcuating Total Fatalities and
Injuries}\label{calcuating-total-fatalities-and-injuries}

\begin{Shaded}
\begin{Highlighting}[]
\NormalTok{totalInjuriesDT <-}\StringTok{ }\NormalTok{stormDT[, .(}\DataTypeTok{FATALITIES =} \KeywordTok{sum}\NormalTok{(FATALITIES), }\DataTypeTok{INJURIES =} \KeywordTok{sum}\NormalTok{(INJURIES), }\DataTypeTok{totals =} \KeywordTok{sum}\NormalTok{(FATALITIES) }\OperatorTok{+}\StringTok{ }\KeywordTok{sum}\NormalTok{(INJURIES)), by =}\StringTok{ }\NormalTok{.(EVTYPE)]}
\NormalTok{totalInjuriesDT <-}\StringTok{ }\NormalTok{totalInjuriesDT[}\KeywordTok{order}\NormalTok{(}\OperatorTok{-}\NormalTok{FATALITIES), ]}
\NormalTok{totalInjuriesDT <-}\StringTok{ }\NormalTok{totalInjuriesDT[}\DecValTok{1}\OperatorTok{:}\DecValTok{10}\NormalTok{, ]}
\KeywordTok{head}\NormalTok{(totalInjuriesDT, }\DecValTok{5}\NormalTok{)}
\end{Highlighting}
\end{Shaded}

\begin{verbatim}
##            EVTYPE FATALITIES INJURIES totals
## 1:        TORNADO       5633    91346  96979
## 2: EXCESSIVE HEAT       1903     6525   8428
## 3:    FLASH FLOOD        978     1777   2755
## 4:           HEAT        937     2100   3037
## 5:      LIGHTNING        816     5230   6046
\end{verbatim}

\subsection{3: Results}\label{results}

\subsubsection{3.1: Events that are Most Harmful to Population
Health}\label{events-that-are-most-harmful-to-population-health}

Melting data.table so that it is easier to put in bar graph format

\begin{Shaded}
\begin{Highlighting}[]
\NormalTok{bad_stuff <-}\StringTok{ }\KeywordTok{melt}\NormalTok{(totalInjuriesDT, }\DataTypeTok{id.vars=}\StringTok{"EVTYPE"}\NormalTok{, }\DataTypeTok{variable.name =} \StringTok{"bad_thing"}\NormalTok{)}
\KeywordTok{head}\NormalTok{(bad_stuff, }\DecValTok{5}\NormalTok{)}
\end{Highlighting}
\end{Shaded}

\begin{verbatim}
##            EVTYPE  bad_thing value
## 1:        TORNADO FATALITIES  5633
## 2: EXCESSIVE HEAT FATALITIES  1903
## 3:    FLASH FLOOD FATALITIES   978
## 4:           HEAT FATALITIES   937
## 5:      LIGHTNING FATALITIES   816
\end{verbatim}

\begin{Shaded}
\begin{Highlighting}[]
\CommentTok{# Create chart}
\NormalTok{healthChart <-}\StringTok{ }\KeywordTok{ggplot}\NormalTok{(bad_stuff, }\KeywordTok{aes}\NormalTok{(}\DataTypeTok{x=}\KeywordTok{reorder}\NormalTok{(EVTYPE, }\OperatorTok{-}\NormalTok{value), }\DataTypeTok{y=}\NormalTok{value))}
\CommentTok{# Plot data as bar chart}
\NormalTok{healthChart =}\StringTok{ }\NormalTok{healthChart }\OperatorTok{+}\StringTok{ }\KeywordTok{geom_bar}\NormalTok{(}\DataTypeTok{stat=}\StringTok{"identity"}\NormalTok{, }\KeywordTok{aes}\NormalTok{(}\DataTypeTok{fill=}\NormalTok{bad_thing), }\DataTypeTok{position=}\StringTok{"dodge"}\NormalTok{)}
\CommentTok{# Format y-axis scale and set y-axis label}
\NormalTok{healthChart =}\StringTok{ }\NormalTok{healthChart }\OperatorTok{+}\StringTok{ }\KeywordTok{ylab}\NormalTok{(}\StringTok{"Frequency Count"}\NormalTok{) }
\CommentTok{# Set x-axis label}
\NormalTok{healthChart =}\StringTok{ }\NormalTok{healthChart }\OperatorTok{+}\StringTok{ }\KeywordTok{xlab}\NormalTok{(}\StringTok{"Event Type"}\NormalTok{) }
\CommentTok{# Rotate x-axis tick labels }
\NormalTok{healthChart =}\StringTok{ }\NormalTok{healthChart }\OperatorTok{+}\StringTok{ }\KeywordTok{theme}\NormalTok{(}\DataTypeTok{axis.text.x =} \KeywordTok{element_text}\NormalTok{(}\DataTypeTok{angle=}\DecValTok{45}\NormalTok{, }\DataTypeTok{hjust=}\DecValTok{1}\NormalTok{))}
\CommentTok{# Set chart title and center it}
\NormalTok{healthChart =}\StringTok{ }\NormalTok{healthChart }\OperatorTok{+}\StringTok{ }\KeywordTok{ggtitle}\NormalTok{(}\StringTok{"Top 10 US Killers"}\NormalTok{) }\OperatorTok{+}\StringTok{ }\KeywordTok{theme}\NormalTok{(}\DataTypeTok{plot.title =} \KeywordTok{element_text}\NormalTok{(}\DataTypeTok{hjust =} \FloatTok{0.5}\NormalTok{))}
\NormalTok{healthChart}
\end{Highlighting}
\end{Shaded}

\includegraphics{Rp_DataProject2_files/figure-latex/healthChart-1.pdf}

\subsubsection{3.2: Events that have the Greatest Economic
Consequences}\label{events-that-have-the-greatest-economic-consequences}

Melting data.table so that it is easier to put in bar graph format

\begin{Shaded}
\begin{Highlighting}[]
\NormalTok{econ_consequences <-}\StringTok{ }\KeywordTok{melt}\NormalTok{(totalCostDT, }\DataTypeTok{id.vars=}\StringTok{"EVTYPE"}\NormalTok{, }\DataTypeTok{variable.name =} \StringTok{"Damage_Type"}\NormalTok{)}
\KeywordTok{head}\NormalTok{(econ_consequences, }\DecValTok{5}\NormalTok{)}
\end{Highlighting}
\end{Shaded}

\begin{verbatim}
##               EVTYPE Damage_Type        value
## 1:             FLOOD    propCost 144657709807
## 2: HURRICANE/TYPHOON    propCost  69305840000
## 3:           TORNADO    propCost  56947380677
## 4:       STORM SURGE    propCost  43323536000
## 5:              HAIL    propCost  15735267513
\end{verbatim}

\begin{Shaded}
\begin{Highlighting}[]
\CommentTok{# Create chart}
\NormalTok{econChart <-}\StringTok{ }\KeywordTok{ggplot}\NormalTok{(econ_consequences, }\KeywordTok{aes}\NormalTok{(}\DataTypeTok{x=}\KeywordTok{reorder}\NormalTok{(EVTYPE, }\OperatorTok{-}\NormalTok{value), }\DataTypeTok{y=}\NormalTok{value))}
\CommentTok{# Plot data as bar chart}
\NormalTok{econChart =}\StringTok{ }\NormalTok{econChart }\OperatorTok{+}\StringTok{ }\KeywordTok{geom_bar}\NormalTok{(}\DataTypeTok{stat=}\StringTok{"identity"}\NormalTok{, }\KeywordTok{aes}\NormalTok{(}\DataTypeTok{fill=}\NormalTok{Damage_Type), }\DataTypeTok{position=}\StringTok{"dodge"}\NormalTok{)}
\CommentTok{# Format y-axis scale and set y-axis label}
\NormalTok{econChart =}\StringTok{ }\NormalTok{econChart }\OperatorTok{+}\StringTok{ }\KeywordTok{ylab}\NormalTok{(}\StringTok{"Cost (dollars)"}\NormalTok{) }
\CommentTok{# Set x-axis label}
\NormalTok{econChart =}\StringTok{ }\NormalTok{econChart }\OperatorTok{+}\StringTok{ }\KeywordTok{xlab}\NormalTok{(}\StringTok{"Event Type"}\NormalTok{) }
\CommentTok{# Rotate x-axis tick labels }
\NormalTok{econChart =}\StringTok{ }\NormalTok{econChart }\OperatorTok{+}\StringTok{ }\KeywordTok{theme}\NormalTok{(}\DataTypeTok{axis.text.x =} \KeywordTok{element_text}\NormalTok{(}\DataTypeTok{angle=}\DecValTok{45}\NormalTok{, }\DataTypeTok{hjust=}\DecValTok{1}\NormalTok{))}
\CommentTok{# Set chart title and center it}
\NormalTok{econChart =}\StringTok{ }\NormalTok{econChart }\OperatorTok{+}\StringTok{ }\KeywordTok{ggtitle}\NormalTok{(}\StringTok{"Top 10 US Storm Events causing Economic Consequences"}\NormalTok{) }\OperatorTok{+}\StringTok{ }\KeywordTok{theme}\NormalTok{(}\DataTypeTok{plot.title =} \KeywordTok{element_text}\NormalTok{(}\DataTypeTok{hjust =} \FloatTok{0.5}\NormalTok{))}
\NormalTok{econChart}
\end{Highlighting}
\end{Shaded}

\includegraphics{Rp_DataProject2_files/figure-latex/econChart-1.pdf}


\end{document}
